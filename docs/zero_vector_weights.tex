\documentclass{amsart}
\usepackage{listings}
\newcommand{\dd}[1]{\frac{\mathrm{d}}{\mathrm{d}#1}}
\newcommand{\PT}{\mathrm{PT}}
\newcommand{\DT}{\mathrm{DT}}
\newcommand{\PPT}[1]{\mathbb{P}^{#1}_{PT}}
\newcommand{\PDT}[1]{\mathbb{P}^{#1}_{DT}}
\newcommand{\NN}{\mathbb{N}}
\title{Weights of Plane Partitions for $\vec{k} = 0^n$}
\author{James Hagborg}
\begin{document}
\maketitle
For fixed $a + b + c = 0$, and $m \in \NN$, define a function on plane
partitions $\pi$ by
\[
  w_m(\pi) = \sum_{(i, j, k) \in \pi} (ai)^m + (bj)^m + (ck)^m
\]
In particular, note that for $m = 0$, $w_0(\pi) = 3|\pi|$.  This holds
for skew or reverse plane partitions.  Next, for
$\vec{k} = (k_1, \cdots, k_n)$, define
\[
  w_{\vec k}(\pi) = \prod_{i=1}^n w_{k_i}(\pi)
\]
If $k_1 = 0$, define $\vec k' = (k_2, k_3, \cdots, k_n)$, so that
$w_{\vec k}(\pi) = 3|\pi|w_{\vec k'}(\pi)$.

Define the following series
\[
  \PT_\lambda(\vec k,q) = \sum_{\pi\in\PPT\lambda} w_{\vec k}(\pi)q^{|\pi|}
\]
\[
  \DT_\lambda(\vec k,q) = \sum_{\pi\in\PDT\lambda} w_{\vec k}(\pi)q^{|\pi|}
\]
Applying our observations about when $k_1 = 0$, we have
\[
  \PT_\lambda(\vec k, q)
  = \sum_{\pi\in\PPT\lambda} 3|\pi|w_{\vec k'}(\pi)q^{|\pi|}
  = 3q\dd{q} \PT_\lambda(\vec k', q)
\]
And similarly
\[
  \DT_\lambda(\vec k, q) = 3q\dd{q} \DT_\lambda(\vec k', q)
\]
Using this rule, we can generalize our formula for $\DT_\lambda$ to a
formula for $\DT_\lambda(0^n)$, by repeatedly applying $3q\dd{q}$, and
using the product rule:
\begin{align*}
  \DT_\lambda(q)
  &= \DT_\emptyset(q)\PT_\lambda(q)
\end{align*}
\begin{align*}
  \DT_\lambda(0, q)
  &= \left(3q\dd{q}\DT_\emptyset(q)\right)\PT_\lambda(q) +
    \DT_\emptyset(q)\left(3q\dd{q}\PT_\lambda(q)\right) \\
  &= \DT_\emptyset(0, q)\PT_\lambda(q) +
    \DT_\emptyset(q)\PT_\lambda(0, q)
\end{align*}
\[
  \DT_\lambda(0^2, q) =
  \DT_\emptyset(0^2, q)\PT_\lambda(q) +
  2\DT_\emptyset(0, q)\PT_\lambda(0, q) +
  \DT_\emptyset(q)PT_\lambda(0^2, q)
\]
And so on.  In general, we have
\[
  \DT_\lambda(0^n, q) =
  \sum_{k = 0}^n \binom{n}{k}
  \DT_\emptyset(0^k, q)\PT_\lambda(0^{n-k}, q)
\]
We have verified this identity in sage, using the following method:
\begin{lstlisting}[language=python]
def check_formula(n, shape, prec=6):
    R.<a,b,c> = ZZ[]
    DT1 = weighted_sum((a, b, c), [0] * n, shape, 'dt', prec)
    DT2 = sum(k *
              weighted_sum((a, b, c), [0] * i, [], 'dt', prec) *
              weighted_sum((a, b, c), [0] * j, shape, 'pt', prec)
              for (i, j), k in binomial_coefficients(n).items())
    assert DT1 == DT2
\end{lstlisting}

This identity has a combinatorial proof that extends the proof of the
identity
\[
  \DT_\lambda(q) = \DT_\emptyset(q)\PT_\lambda(q)
\]
One can think of the weight $w_{0^n}(\pi) = 3^n|\pi|^n$ as counting
the number of ways to pick $s_1, \cdots, s_n \in D(\pi) \times
\{x,y,z\}$.  That is, each $s_i$ is a block in the diagram of $\pi$,
along with a choice of axis.  Thus, we can think of $\DT_\lambda(0^n,
q)$ as counting the number of plane partitions with $n$ ``special
points,'' and similarly for $\PT_\lambda(0^n, q)$.

The bijection works as follows.  Start with a plane partition
$\pi \in \PDT{\lambda}$, and
$s_1, \cdots, s_n \in D(\pi) \times \{x, y, z\}$.  This corresponds to
a pair of $\mu \in \PDT{\emptyset}$ and $\nu \in \PPT{\lambda}$, with
$|\pi| = |\mu| + |\nu|$.  Thus, there exists a bijection between the
cells of $D(\pi)$ and $D(\mu) \sqcup D(\nu)$, so we have some $s_1',
\cdots s_n' \in D(\mu) \sqcup D(\nu)$.  In other words, we can extend
our bijection to keep track of where the ``special points'' lie.

For any given $k$, there are $\binom{n}{k}$ ways to have $k$ of the
points land in $D(\mu)$ and the rest land in $D(\nu)$.  Thus,
\[
  \DT_\lambda(0^n, q) =
  \sum_{k = 0}^n \binom{n}{k}
  \DT_\emptyset(0^k, q)\PT_\lambda(0^{n-k}, q)
\]

In general, if $\vec k$ contains non-zero elements, this bijection no
longer works.  However, we can split the weight function up as
\[
  w_{\vec k}(\pi) = \sum_{s_1, \cdots s_n \in S} \prod_{i=1}^n
  W_{s_i}^{k_i}
\]
where $S = D(\pi) \times \{x, y, z\}$ and
\[
  W_{(i, j, k), d} = \begin{cases}
    ai, &\quad d = x \\
    bj, &\quad d = y \\
    ck, &\quad d = z
  \end{cases}
\]
It may be hard to reason about what this is counting, especially for
when $k_i$ is odd and $a, b, c$ may be negative.
\end{document}