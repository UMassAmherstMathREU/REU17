\documentclass{report}
\begin{document}
  \begin{paragraph}
    {}
  In order to map outer partitions in P_{DT}^{\lambda} of size n into pairs of normal partitions in P_{DT}^{\emptyset} and P_{PT}^{\lambda}
  whose sum of sizes is n, we first use the Hillman Grassl algorithm on each partition giving Hillman Grassl tables,
  transforming the partitions as necessary. Then we use an arbitrary bijection between the resulting Hillman Grassl tables
  to map the table from the P_{DT}^{\lambda} side to the \left(P_{DT}^{\emptyset}, P_{PT}^{\lambda}\right) side. By combining these
  bijections we can map a partition on the DT side into an HG data table, then map that into a pair of HG tables on the PT side,
  and finally map that pair back into a pair of partitions on the PT side, thus providing an overall bijection between the two
  sides. This proves combinatorially the correspondence between the PT side and the DT side.
\end{paragraph}
  
\end{document}
