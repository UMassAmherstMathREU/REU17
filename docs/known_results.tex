\documentclass{amsart}
\usepackage{amsthm}
\newtheorem{defn}{Definition}
\newtheorem{conj}{Conjecture}
\newcommand{\pt}[1]{\mathbb{P}_\mathrm{PT}^{#1}}
\newcommand{\dt}[1]{\mathbb{P}_\mathrm{DT}^{#1}}
\newcommand{\ZZ}{\mathbb{Z}}
\newcommand{\NN}{\mathbb{N}}
\newcommand{\HG}[1]{\mathrm{HG}_{#1}}
\newcommand{\hg}[1]{\mathrm{hg}_{#1}}
\newcommand{\Rem}{\mathrm{Rem}}
\newcommand{\ch}{\mathrm{ch}}
\newcommand{\dd}[1]{\frac{d}{d #1}}
\begin{document}
\section{PT/DT Correspondence with one insertion}
\begin{defn}
  Fix $t_1 + t_2 + t_3 = 0$.  Let $k_1, \cdots, k_n \in \NN$.
  Define
  \[
    PT_\lambda(k_1, \cdots, k_n; q)
    = \sum_{\pi\in\pt{\lambda}} (-q)^{|\pi|}
    \prod_{i = 1}^n \ch_{k_i}(\pi)
  \]
  \[
    DT_\lambda(k_1, \cdots, k_n; q)
    = \sum_{\pi\in\dt{\lambda}} (-q)^{|\pi|}
    \prod_{i = 1}^n \ch_{k_i}(\pi)
  \]
  \[
    DT_\lambda'(k_1, \cdots, k_n; q)
    = \frac{DT_\lambda(k_1, \cdots, k_n; q)}{DT_\emptyset(q)}
  \]
  where $\ch_k$ is defined by the series
  \[
    \sum_{k = 0}^\infty \ch_k(\pi)z^k =
    1 - \prod_{i=1}^3\left(1 - e^{t_i z}\right)
    \sum_{\square\in\pi} e^{T(\square)z}
  \]
  with $T(i, j, k) = t_1i + t_2j + t_3k$.
\end{defn}
It will be useful to handle all values of $k$ at once by adding a new
variable:
\begin{defn}
  \[ PT_\lambda(z, q) = \sum_{k \geq 0} PT(k; q)z^k \]
  \[ DT'_\lambda(z, q) = \sum_{k \geq 0} DT'(k; q)z^k \]
\end{defn}
\begin{conj}
  For any partition $\lambda$,
  \[ DT'_\lambda(z, q) = DT'_\emptyset(z, q)PT_\lambda(z, q) \]
  Or more explicitely,
  \[ DT'_\lambda(k; q) =
    \sum_{l = 0}^k DT'_\emptyset(l; q)PT_\lambda(k - l; q) \]
\end{conj}
This was checked using the SageMath code in \texttt{chern\_char.sage}.

\section{Two Insertions}
For the two insertion case, the formula is not quite as simple.  We
can still define the series
\[
  PT_\lambda(z_1, z_2, q) =
  \sum_{k_1, k_2 \geq 0} PT_\lambda(k_1, k_2; q) z_1^{k_1} z_2^{k_2}
\]
\[
  DT_\lambda'(z_1, z_2, q) =
  \sum_{k_1, k_2 \geq 0} DT_\lambda'(k_1, k_2; q) z_1^{k_1} z_2^{k_2}
\]

In this case, we have some remainder term:
\[
  DT_\lambda'(z_1, z_2, q)
  = DT_\emptyset'(z_1, z_2, q) PT_\lambda(z_1, z_2, q)
  + t_1t_2t_3 Rem(z_1, z_2, q)
\]
where
\[
  Rem(z_1, z_2, q) = \sum_{k_1, k_2} Rem(k_1, k_2; q)z_1^{k_1}z_2^{k_2}
\]
For $\min(k_1, k_2) \leq 3$, $Rem(k_1, k_2; q) = 0$.  When $k_1, k_2 \geq 4$, we have more interesting behavior.

\subsection{For $k_1=4$}
In this case, we have
\[
  [z_1^4] Rem(z_1, z_2, q) =
  \left(q\dd q DT'_\emptyset(z_2, q)\right)
  \left(\dd{z_2} PT_\lambda(z_2, q)\right)
\]
This is verified by the sage function \texttt{check\_4\_n\_formula}.

\subsection{For $k_1=5$}
Here, we have
\begin{align*}
  [z_1^5] Rem(z_1, z_2, q) =
  &\frac{1}{2}\left(\dd{z_2} - \frac{P'}{P}\right)
  \left[
    \left(q\dd q DT'_\emptyset(z_2, q)\right)
    \left(\dd{z_2} PT_\lambda(z_2, q)\right)
  \right] \\
  &- \left(q\dd q DT'_\emptyset(z_2, q)\right)PT_\lambda(z_2, 2, q)
\end{align*}
Where
\[
  P = \prod_{i=1}^3 (1 - e^{t_i})
\]

\subsection{For $k_1 \geq 6$}
I can explicitely compute $(6,6)$, but after $(7,6)$, the computation
fails with the current setup.

The coefficients for $(6,6)$ look like they include a term of the form
\[
  \frac{1}{6}\left(\dd z_2\right)^2 \left[
    \left(q\dd q DT'_\emptyset(z_2, q)\right)
    \left(\dd{z_2}PT_\lambda(z_2, q)\right)\right]
\]
This suggests that the term $1/n! (d/dz_2)^{n-1}$ always appears.  If
we re-write the derivative on the second term using the product rule,
we can get a term $1/n! (d/dz_2)^{n}$, which suggests something like
the substitution $z \leftarrow z_1 + z_2$ to handle these terms.  This
may be related to the $P'/P$ that appears.

In $(6,6)$, terms involving $PT(\cdot, 3)$ appear, suggesting that at
most $PT(k_1-3, k_2-3)$ appears when there are two insertions on the
PT side.

Possibly, we need to add something of the form $(t_1t_2t_3)^2$, with a
second derivative on the DT side, and so on.  I would try the
computation modulo $(t_1t_2t_3)^2$, but it seems that the remainder is
divisible by $t_1t_2t_3$, so this would always give zero remainder.
Maybe I should try modulo $(t_1t_2t_3)^3$.

\section{Equivarient Vertex Weight}
When we remove the assumption $t_1 + t_2 + t_3 = 0$ and add the vertex
weight to the calculation, the formula for one insertion still holds.
I haven't checked the partial result we have for two insertions.
\end{document}