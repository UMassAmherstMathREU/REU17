\documentclass{amsart}
\usepackage{amsthm}
\newcommand{\pt}[1]{\mathbb{P}_\mathrm{PT}^{#1}}
\newcommand{\dt}[1]{\mathbb{P}_\mathrm{DT}^{#1}}
\newcommand{\ZZ}{\mathbb{Z}}
\newcommand{\NN}{\mathbb{N}}
\newcommand{\HG}[1]{\mathrm{HG}_{#1}}
\newcommand{\hg}[1]{\mathrm{hg}_{#1}}
\newtheorem{definition}{Definition}
\newtheorem{theorem}{Theorem}
\begin{document}
First, some basic notation:

\begin{definition}
  A \emph{weak reverse plane partition} with shape $\lambda$ is a
  tableau of shape $\lambda$ containing natural numbers (including 0),
  such that the rows and columns are weakly increasing.
  
  Let $\pt{\lambda}$ be the set of all weak reverse plane partitions
  with shape $\lambda$.
\end{definition}
\begin{definition}
  A plane partition with \emph{shape at infinity} $\lambda$, or
  \emph{skew plane partition}, is a tableau containing entries from
  $\ZZ_+ \cup \{\infty\}$, such that rows and columns are weakly
  decreasing, and a cell contains $\infty$ iff it is contained in
  $\lambda$.
  
  Let $\dt{\lambda}$ be the set of all plane partitions with the shape
  at infinity $\lambda$.  In particular, $\dt{\emptyset}$ is the set of
  all plane partitions without any shape at infinity.
\end{definition}
\begin{definition}
  If $\pi \in \pt{\lambda}$, define the \emph{size} of $\pi$, written
  $|\pi|$, to be the sum of all entries in $\pi$.  Similarly, if
  $|\pi| \in \dt{\lambda}$, define $|\pi|$ to be the sum of all
  non-infinity entries in $\pi$.
\end{definition}

Given $\lambda$, we wish to find a bijection
$\Psi : \pt{\lambda} \times \dt{\emptyset} \to \dt{\lambda}$, such
that $|\Psi(\mu, \pi) = |\pi| + |\mu|$.  In order to do this, we make
use of the Hillman-Grassl Correspondence, which associates to all
$\pi \in \pt{\lambda}$ a function $f : \lambda \to \NN$ (where
$\lambda$ is identified with its diagram, as a subset of $\ZZ_+^2$).
The size of $\pi$ is encoded as a weighted sum over values of $f$.  We
also adapt this correspondence to work on $\dt{\lambda}$, giving a
function with domain $\lambda^* = \ZZ_+^2 \setminus \lambda$.  In
particular, for $\dt{\emptyset}$, the domain is just $\ZZ_+^2$.  Thus
if we can find a bijection between $\lambda^*$ and the disjoint union
$\lambda \sqcup \ZZ_+^2$ which respects the weights of each cell, this
will yield a bijection between Hillman-Grassl functions, which in turn
gives the bijection $\Psi$ we are looking for.

To make this more precise, we have the following definitions:
\begin{definition}
  A \emph{Hillman-Grassl tableau} or \emph{HG-tableau} of shape
  $\lambda$ is a tableau with entries from $\NN$, or equivalently a
  function $f : \lambda \to \NN$.

  Let $\HG{\lambda}$ denote the set of all HG-tableaux, and let
  $\hg{\lambda} : \pt{\lambda} \to \HG{\lambda}$ denote the
  Hillman-Grassl algorithm.
\end{definition}
\begin{definition}
  Given a partition $\lambda$ and a cell $u \in \lambda$, the
  \emph{hook length} $h_\lambda(u)$ is the number of cells in lambda
  directly below or to the right of $u$, including $u$.  That is,
  \[
    h_\lambda(u) = \lambda_i - j + \lambda_j' - i + 1
  \]
\end{definition}
\begin{definition}
  The \emph{size} of an HG-tableau $f \in \HG{\lambda}$ is the
  weighted sum over hook lengths
  \[
    |f| = \sum_{u \in \lambda} h_\lambda(u) f(u)
  \]
\end{definition}
\begin{theorem}
  The 
\end{theorem}
\end{document}