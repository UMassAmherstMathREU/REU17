\documentclass{amsart}
\usepackage{amsthm}
\newcommand{\pt}[1]{\mathbb{P}_\mathrm{PT}^{#1}}
\newcommand{\dt}[1]{\mathbb{P}_\mathrm{DT}^{#1}}
\newcommand{\ZZ}{\mathbb{Z}}
\newcommand{\NN}{\mathbb{N}}
\newcommand{\HG}[1]{\mathrm{HG}_{#1}}
\newcommand{\hg}[1]{\mathrm{hg}_{#1}}
\newtheorem{theorem}{Theorem}
\newtheorem{corollary}{Corollary}
\theoremstyle{definition}
\newtheorem{definition}{Definition}
\newtheorem{example}{Example}
\begin{document}
\section{Perliminaries and the Hillman-Grassl Algorithm}
First, some basic notation:

\begin{definition}
  A \emph{weak reverse plane partition} with shape $\lambda$ is a
  tableau of shape $\lambda$ containing natural numbers (including 0),
  such that the rows and columns are weakly increasing.
  
  Let $\pt{\lambda}$ be the set of all weak reverse plane partitions
  with shape $\lambda$.
\end{definition}
\begin{definition}
  A plane partition with \emph{shape at infinity} $\lambda$, or
  \emph{skew plane partition}, is a tableau containing entries from
  $\ZZ_+ \cup \{\infty\}$, such that rows and columns are weakly
  decreasing, and a cell contains $\infty$ iff it is contained in
  $\lambda$.
  
  Let $\dt{\lambda}$ be the set of all plane partitions with the shape
  at infinity $\lambda$.  In particular, $\dt{\emptyset}$ is the set of
  all plane partitions without any shape at infinity.
\end{definition}
\begin{definition}
  If $\pi \in \pt{\lambda}$, define the \emph{size} of $\pi$, written
  $|\pi|$, to be the sum of all entries in $\pi$.  Similarly, if
  $|\pi| \in \dt{\lambda}$, define $|\pi|$ to be the sum of all
  non-infinity entries in $\pi$.
\end{definition}

Given $\lambda$, we wish to find a bijection
$\Psi : \pt{\lambda} \times \dt{\emptyset} \to \dt{\lambda}$, such
that $|\Psi(\mu, \pi) = |\pi| + |\mu|$.  In order to do this, we make
use of the Hillman-Grassl Correspondence, which associates to all
$\pi \in \pt{\lambda}$ a function $f : \lambda \to \NN$ (where
$\lambda$ is identified with its diagram, as a subset of $\ZZ_+^2$).
The size of $\pi$ is encoded as a weighted sum over values of $f$.  We
also adapt this correspondence to work on $\dt{\lambda}$, giving a
function with domain $\lambda^* = \ZZ_+^2 \setminus \lambda$.  In
particular, for $\dt{\emptyset}$, the domain is just $\ZZ_+^2$.  Thus
if we can find a bijection between $\lambda^*$ and the disjoint union
$\lambda \sqcup \ZZ_+^2$ which respects the weights of each cell, this
will yield a bijection between Hillman-Grassl functions, which in turn
gives the bijection $\Psi$ we are looking for.

To make this more precise, we have the following definitions:
\begin{definition}
  A \emph{Hillman-Grassl tableau} or \emph{HG-tableau} of shape
  $\lambda$ is a tableau with entries from $\NN$, or equivalently a
  function $f : \lambda \to \NN$.

  Let $\HG{\lambda}$ denote the set of all HG-tableaux, and let
  $\hg{\lambda} : \pt{\lambda} \to \HG{\lambda}$ denote the
  Hillman-Grassl algorithm.
\end{definition}
\begin{definition}
  Given a partition $\lambda$ and a cell $u \in \lambda$, the
  \emph{hook length} $h_\lambda(u)$ is the number of cells in $\lambda$
  directly below or to the right of $u$, including $u$.  That is,
  \[
    h_\lambda(u) = \lambda_i - j + \lambda_j' - i + 1
  \]
\end{definition}
\begin{definition}
  The \emph{size} of an HG-tableau $f \in \HG{\lambda}$ is the
  weighted sum over hook lengths
  \[
    |f| = \sum_{u \in \lambda} h_\lambda(u) f(u)
  \]
\end{definition}
With these definitions, we can write the defining property of the
Hillman-Grassl algorithm as follows:
\begin{theorem}
  For all $\pi \in \pt{\lambda}$, $|\hg{\lambda}(\pi)| = |\pi|$.
\end{theorem}

\section{The modified Hillman-Grassl Algorthim}

We now define a modified version of the Hillman-Grassl algorithm which
operates on $\dt{\lambda}$.
\begin{enumerate}
\item Given $\pi \in \dt{\lambda}$, embed $\pi$ in a rectangular
  tableau of shape $R$ by placing zeroes in the remaining cells.
\item Reverse the rows and columns of $\pi$, and remove all
  cells containing $\infty$ (which are at the bottom-right at this
  point), to form a weak reverse plane partition $\pi'$.
\item Run the Hillman-Grassl algorithm on $\pi'$, to create a
  HG-tableau $f'$.
\item Reverse the rows and columns of $f'$, to create a skew tableau
  $f$, with shape $R \setminus \lambda$.  Extend this to a function
  on all of $\lambda^* = \ZZ_+^2 \setminus \lambda$ by defining $f(u)
  = 0$ if $u \notin R$.
\end{enumerate}

Before we go into the properties of this algorithm, let's see an
example.
\begin{example}
  Example.
\end{example}

There are a few thing we need to verify about this algorithm.  Fisrt,
we have some freedom in our choice of which rectangle $R$ to embed
in.  However, since the Hillman-Grassl algorithm operates only on
non-zero entries, enlarging $R$, which in turn pads $\pi'$ with
zeroes, has only the effect of padding $f$ with zeroes.  Since we
extend $f$ by adding zeroes anyway, this has no effect.

Second, note that although the domain of $f$ is now infinite, $f$
vanishes at all but finitely many points, so we can still define
\[ |f| = \sum_{u \in \lambda^*} h_\lambda^*(u) f(u) \]
where $h_\lambda^*$ denotes the \emph{reverse hook length}, or
\[ h_\lambda^*(i, j) = j - \lambda_i + i - \lambda_j' - 1. \]
In other words, the reverse hook lenght is the number of cells above
and to the left of $u$, including $u$, but not in $\lambda$.  This
corresponds to the hook length of a cell in $f'$, before reversing
rows and columns.

Let $\HG{\lambda}^*$ be the set of functions $f : \lambda^* \to
\NN$ with finitely many non-zero entries, and let $\hg{\lambda}^* :
\dt{\lambda} \to \HG{\lambda}^*$ be the modified Hillman-Grassl
algorithm.  From our observation above, we have
\[
  |\hg{\lambda}^*(\pi)| = |\pi|
\]
for any $\pi \in \dt{\lambda}$.

\subsection{The McMahon Formula Revisited}
Using this modified algorithm, we immediately have the following:
\begin{theorem}
  Let $\lambda$ be a partition.  Then
  \[
    \sum_{\pi \in \dt{\lambda}} q^{|\pi|} =
    \prod_{u \in \lambda^*} \frac{1}{1 - q^{h_\lambda^*(u)}}
  \]
\end{theorem}
\begin{proof}
  Picking a term $q^\alpha$ in the expansion of the right-hand side is
  equivalent to picking values of $f(u)$, such that $\sum
  h_\lambda^*(u)f(u) = \alpha$.
\end{proof}

For comparison, Theorem 7.22.1 in Stanley, which is proved using the
original Hillman-Grassl algorithm, states
\[
  \sum_{\pi \in \pt{\lambda}} q^{|\pi|} =
  \prod_{u \in \lambda} \frac{1}{1 - q^{h_\lambda(u)}}
\]

If we recall that $\dt{\emptyset}$ is the set of all non-skew plane
partitions, we can consider the McMahon formula to be a special case
of this formula.
\begin{corollary}
\[
  \sum_{\pi \in \dt{\emptyset}} q^{|\pi|} =
  \prod_{k \geq 1} \frac{1}{\left(1 - q^k\right)^k}
\]
\end{corollary}
\begin{proof}
  Since $\lambda = \emptyset$ here, the reverse hook length is
  \[
    h_\emptyset^*(i, j) = i + j - 1
  \]
  so that every hook length $k$ occurs exactly $k$ times.
\end{proof}
\end{document}