\documentclass{amsart}
\usepackage{amsthm}
\newcommand{\pt}[1]{\mathbb{P}_\mathrm{PT}^{#1}}
\newcommand{\dt}[1]{\mathbb{P}_\mathrm{DT}^{#1}}
\newcommand{\ZZ}{\mathbb{Z}}
\newcommand{\NN}{\mathbb{N}}
\newcommand{\HG}[1]{\mathrm{HG}_{#1}}
\newcommand{\hg}[1]{\mathrm{hg}_{#1}}
\newtheorem{theorem}{Theorem}
\newtheorem{lemma}{Lemma}
\newtheorem{corollary}{Corollary}
\theoremstyle{definition}
\newtheorem{definition}{Definition}
\newtheorem{example}{Example}
\begin{document}
\section{Preliminaries and the Hillman-Grassl Algorithm}
First, some basic notation:

\begin{definition}
  A \emph{weak reverse plane partition} with shape $\lambda$ is a
  tableau of shape $\lambda$ containing natural numbers (including 0),
  such that the rows and columns are weakly increasing.
  
  Let $\pt{\lambda}$ be the set of all weak reverse plane partitions
  with shape $\lambda$.
\end{definition}
\begin{definition}
  A plane partition with \emph{shape at infinity} $\lambda$, or
  \emph{skew plane partition}, is a tableau containing entries from
  $\ZZ_+ \cup \{\infty\}$, such that rows and columns are weakly
  decreasing, and a cell contains $\infty$ iff it is contained in
  $\lambda$.
  
  Let $\dt{\lambda}$ be the set of all plane partitions with the shape
  at infinity $\lambda$.  In particular, $\dt{\emptyset}$ is the set of
  all plane partitions without any shape at infinity.
\end{definition}
\begin{definition}
  If $\pi \in \pt{\lambda}$, define the \emph{size} of $\pi$, written
  $|\pi|$, to be the sum of all entries in $\pi$.  Similarly, if
  $|\pi| \in \dt{\lambda}$, define $|\pi|$ to be the sum of all
  non-infinity entries in $\pi$.
\end{definition}

Given $\lambda$, we wish to find a bijection
$\Psi : \pt{\lambda} \times \dt{\emptyset} \to \dt{\lambda}$, such
that $|\Psi(\mu, \pi) = |\pi| + |\mu|$.  In order to do this, we make
use of the Hillman-Grassl Correspondence, which associates to all
$\pi \in \pt{\lambda}$ a function $f : \lambda \to \NN$ (where
$\lambda$ is identified with its diagram, as a subset of $\ZZ_+^2$).
The size of $\pi$ is encoded as a weighted sum over values of $f$.  We
also adapt this correspondence to work on $\dt{\lambda}$, giving a
function with domain $\lambda^* = \ZZ_+^2 \setminus \lambda$.  In
particular, for $\dt{\emptyset}$, the domain is just $\ZZ_+^2$.  Thus
if we can find a bijection between $\lambda^*$ and the disjoint union
$\lambda \sqcup \ZZ_+^2$ which respects the weights of each cell, this
will yield a bijection between Hillman-Grassl functions, which in turn
gives the bijection $\Psi$ we are looking for.

To make this more precise, we have the following definitions:
\begin{definition}
  A \emph{Hillman-Grassl tableau} or \emph{HG-tableau} of shape
  $\lambda$ is a tableau with entries from $\NN$, or equivalently a
  function $f : \lambda \to \NN$.

  Let $\HG{\lambda}$ denote the set of all HG-tableaux, and let
  $\hg{\lambda} : \pt{\lambda} \to \HG{\lambda}$ denote the
  Hillman-Grassl algorithm.
\end{definition}
\begin{definition}
  Given a partition $\lambda$ and a cell $u \in \lambda$, the
  \emph{hook length} $h_\lambda(u)$ is the number of cells in $\lambda$
  directly below or to the right of $u$, including $u$.  That is,
  \[
    h_\lambda(u) = \lambda_i - j + \lambda_j' - i + 1
  \]
\end{definition}
\begin{definition}
  The \emph{size} of an HG-tableau $f \in \HG{\lambda}$ is the
  weighted sum over hook lengths
  \[
    |f| = \sum_{u \in \lambda} h_\lambda(u) f(u)
  \]
\end{definition}
With these definitions, we can write the defining property of the
Hillman-Grassl algorithm as follows:
\begin{theorem}
  For all $\pi \in \pt{\lambda}$, $|\hg{\lambda}(\pi)| = |\pi|$.
\end{theorem}

\section{The modified Hillman-Grassl Algorthim}

We now define a modified version of the Hillman-Grassl algorithm which
operates on $\dt{\lambda}$.
\begin{enumerate}
\item Given $\pi \in \dt{\lambda}$, embed $\pi$ in a rectangular
  tableau of shape $R$ by placing zeroes in the remaining cells.
\item Reverse the rows and columns of $\pi$, and remove all
  cells containing $\infty$ (which are at the bottom-right at this
  point), to form a weak reverse plane partition $\pi'$.
\item Run the Hillman-Grassl algorithm on $\pi'$, to create a
  HG-tableau $f'$.
\item Reverse the rows and columns of $f'$, to create a skew tableau
  $f$, with shape $R \setminus \lambda$.  Extend this to a function
  on all of $\lambda^* = \ZZ_+^2 \setminus \lambda$ by defining $f(u)
  = 0$ if $u \notin R$.
\end{enumerate}

Before we go into the properties of this algorithm, let's see an
example.
\begin{example}
  Let $\lambda = (1)$, and let $\pi$ be
  \[
    {\def\lr#1{\multicolumn{1}{|@{\hspace{.6ex}}c@{\hspace{.6ex}}|}{\raisebox{-.3ex}{$#1$}}}
      \raisebox{-.6ex}{$
        \begin{array}[b]{*{3}c}\cline{1-3}
          \lr{\infty}&\lr{2}&\lr{2}\\\cline{1-3}
          \lr{3}&\lr{1}\\\cline{1-2}
          \lr{1}\\\cline{1-1}
        \end{array}$}
    }
  \]
  We can embed $\pi$ in a $3 \times 3$ rectangle:
  \[
    {\def\lr#1{\multicolumn{1}{|@{\hspace{.6ex}}c@{\hspace{.6ex}}|}{\raisebox{-.3ex}{$#1$}}}
      \raisebox{-.6ex}{$
        \begin{array}[b]{*{3}c}\cline{1-3}
          \lr{\infty}&\lr{2}&\lr{2}\\\cline{1-3}
          \lr{3}&\lr{1}&\lr{0}\\\cline{1-3}
          \lr{1}&\lr{0}&\lr{0}\\\cline{1-3}
        \end{array}$}
    }
  \]
  And form $\pi'$ by flipping rows and columns, and removing $\infty$:
  \[
    {\def\lr#1{\multicolumn{1}{|@{\hspace{.6ex}}c@{\hspace{.6ex}}|}{\raisebox{-.3ex}{$#1$}}}
      \raisebox{-.6ex}{$
        \begin{array}[b]{*{3}c}\cline{1-3}
          \lr{0}&\lr{0}&\lr{1}\\\cline{1-3}
          \lr{0}&\lr{1}&\lr{3}\\\cline{1-3}
          \lr{2}&\lr{2}\\\cline{1-2}
        \end{array}$}
    }
  \]
  We can now run the Hillman-Grassl algorithm, to get a tableau of the
  same shape:
  \[
    {\def\lr#1{\multicolumn{1}{|@{\hspace{.6ex}}c@{\hspace{.6ex}}|}{\raisebox{-.3ex}{$#1$}}}
      \raisebox{-.6ex}{$
        \begin{array}[b]{*{3}c}\cline{1-3}
          \lr{0}&\lr{0}&\lr{1}\\\cline{1-3}
          \lr{1}&\lr{0}&\lr{1}\\\cline{1-3}
          \lr{1}&\lr{0}\\\cline{1-2}
        \end{array}$}
    }
  \]
  We invert the rows and columns back to obtain a skew tableau:
  \[
    {\def\lr#1{\multicolumn{1}{|@{\hspace{.6ex}}c@{\hspace{.6ex}}|}{\raisebox{-.3ex}{$#1$}}}
      \raisebox{-.6ex}{$
        \begin{array}[b]{*{3}c}\cline{2-3}
          &\lr{0}&\lr{1}\\\cline{1-3}
          \lr{1}&\lr{0}&\lr{1}\\\cline{1-3}
          \lr{1}&\lr{0}&\lr{0}\\\cline{1-3}
        \end{array}$}
    }
  \]
  We can consider this a function $f$ from the cells of this tableau,
  indexed by $\ZZ_+^2$ to $\NN$.  We can extend this to a function on
  all $\ZZ_+^2 \setminus \{(1, 1)\}$ by adding zeros, so values of $f$
  are given by
  \[
    {\def\lr#1{\multicolumn{1}{|@{\hspace{.6ex}}c@{\hspace{.6ex}}|}{\raisebox{-.3ex}{$#1$}}}
      \raisebox{-.6ex}{$
        \begin{array}[b]{*{5}c}\cline{2-5}
          &\lr{0}&\lr{1}&\lr{0}&\cdots\\\cline{1-5}
          \lr{1}&\lr{0}&\lr{1}&\lr{0}&\cdots\\\cline{1-5}
          \lr{1}&\lr{0}&\lr{0}&\lr{0}&\cdots\\\cline{1-5}
          \lr{0}&\lr{0}&\lr{0}&\lr{0}&\cdots\\\cline{1-5}
          \lr{\vdots} & \lr{\vdots} & \lr{\vdots} & \lr{\vdots} & \ddots
        \end{array}$}
    }
  \]
\end{example}

There are a few thing we need to verify about this algorithm.  Fisrt,
we have some freedom in our choice of which rectangle $R$ to embed
in.  However, since the Hillman-Grassl algorithm operates only on
non-zero entries, enlarging $R$, which in turn pads $\pi'$ with
zeroes, has only the effect of padding $f$ with zeroes.  Since we
extend $f$ by adding zeroes anyway, this has no effect.

Second, note that although the domain of $f$ is now infinite, $f$
vanishes at all but finitely many points, so we can still define
\[ |f| = \sum_{u \in \lambda^*} h_\lambda^*(u) f(u) \]
where $h_\lambda^*$ denotes the \emph{reverse hook length}, or
\[ h_\lambda^*(i, j) = j - \lambda_i + i - \lambda_j' - 1. \]
In other words, the reverse hook lenght is the number of cells above
and to the left of $u$, including $u$, but not in $\lambda$.  This
corresponds to the hook length of a cell in $f'$, before reversing
rows and columns.

Let $\HG{\lambda}^*$ be the set of functions $f : \lambda^* \to
\NN$ with finitely many non-zero entries, and let $\hg{\lambda}^* :
\dt{\lambda} \to \HG{\lambda}^*$ be the modified Hillman-Grassl
algorithm.  From our observation above, we have
\[
  |\hg{\lambda}^*(\pi)| = |\pi|
\]
for any $\pi \in \dt{\lambda}$.

\begin{example}
  Let's continue with the example above.  First, we compute the
  reverse hook lengths.
  \begin{center}
    \begin{tabular}{c c}
      $f(u)$ & $h_\lambda^*(u)$ \\
      \rule{0pt}{8ex}
      $
      {\def\lr#1{\multicolumn{1}{|@{\hspace{.6ex}}c@{\hspace{.6ex}}|}{\raisebox{-.3ex}{$#1$}}}
      \raisebox{-.6ex}{$
      \begin{array}[b]{*{3}c}\cline{2-3}
        &\lr{0}&\lr{1}\\\cline{1-3}
        \lr{1}&\lr{0}&\lr{1}\\\cline{1-3}
        \lr{1}&\lr{0}&\lr{0}\\\cline{1-3}
      \end{array}$}
      }
      $ & $
          {\def\lr#1{\multicolumn{1}{|@{\hspace{.6ex}}c@{\hspace{.6ex}}|}{\raisebox{-.3ex}{$#1$}}}
          \raisebox{-.6ex}{$
          \begin{array}[b]{*{3}c}\cline{2-3}
            &\lr{1}&\lr{2}\\\cline{1-3}
            \lr{1}&\lr{3}&\lr{4}\\\cline{1-3}
            \lr{2}&\lr{4}&\lr{5}\\\cline{1-3}
          \end{array}$}
      }
      $
    \end{tabular}
  \end{center}
  Now we have
  \[
    |f| = (1)(2) + (1)(1) + (1)(4) + (1)(2) = 9
  \]
  The original plane partition had
  \[
    |\pi| = 2 + 2 + 3 + 1 + 1 = 9
  \]
  as expected.
\end{example}

\subsection{The McMahon Formula Revisited}
Using this modified algorithm, we immediately have the following:
\begin{theorem}
  Let $\lambda$ be a partition.  Then
  \[
    \sum_{\pi \in \dt{\lambda}} q^{|\pi|} =
    \prod_{u \in \lambda^*} \frac{1}{1 - q^{h_\lambda^*(u)}}
  \]
\end{theorem}
\begin{proof}
  Picking a term $q^\alpha$ in the expansion of the right-hand side is
  equivalent to picking values of $f(u)$, such that $\sum
  h_\lambda^*(u)f(u) = \alpha$.
\end{proof}

For comparison, Theorem 7.22.1 in Stanley, which is proved using the
original Hillman-Grassl algorithm, states
\[
  \sum_{\pi \in \pt{\lambda}} q^{|\pi|} =
  \prod_{u \in \lambda} \frac{1}{1 - q^{h_\lambda(u)}}
\]

If we recall that $\dt{\emptyset}$ is the set of all non-skew plane
partitions, we can consider the McMahon formula to be a special case
of this formula.
\begin{corollary}
\[
  \sum_{\pi \in \dt{\emptyset}} q^{|\pi|} =
  \prod_{k \geq 1} \frac{1}{\left(1 - q^k\right)^k}
\]
\end{corollary}
\begin{proof}
  Since $\lambda = \emptyset$ here, the reverse hook length is
  \[
    h_\emptyset^*(i, j) = i + j - 1
  \]
  so that every hook length $k$ occurs exactly $k$ times.
\end{proof}

\section{A Bijection on Hooks}
We have defined bijections $\hg{\lambda} : \pt{\lambda} \to
\HG{\lambda}$ and $\hg{\lambda}^* : \dt{\lambda} \to \HG{\lambda}^*$.
We now want to find a bijection $\HG{\lambda} \times \HG{\emptyset}^*
\to \HG{\lambda}^*$.  Equivalently, we want to prove
\[
  \left(\prod_{u \in \lambda} \frac{1}{1-q^{h_\lambda(u)}}\right)
  \left(\prod_{i \geq 1} \frac{1}{(1-q^k)^k}\right) =
  \prod_{u \in \lambda^*} \frac{1}{1-q^{h_\lambda^*(u)}}
\]
The nicest way to do with would be to directly match up terms in the
product.  This means showing that for all $k \in \ZZ_+$, $\lambda$ has
$k$ more outer $k$-hooks than inner $k$-hooks.  

To make the bijection explicit, we need to find
$\phi: \lambda^* \to \lambda \sqcup \ZZ_+^2$, such that if
$\phi(u) \in \lambda$,
\[ h_\lambda(\phi(u)) = h_\lambda^*(u) \]
and if $\phi(u) \in \ZZ_+^2$,
\[ h_\emptyset^*(\phi(u)) = h_\lambda^*(u) \]
Once we have this, we can define $\Phi : \HG{\lambda} \times
\HG{\emptyset}^* \to \HG{\lambda}^*$ in the natural way, by
\[ 
  [\Phi(f, g)](u) = \begin{cases}
    f(\phi(u)), &\quad \phi(u) \in \lambda \\
    g(\phi(u)), &\quad \phi(u) \in \ZZ_+^2
  \end{cases}
\]
This way, we have
\[
  |\Phi(f, g)| =
  \sum_{u \in \lambda^*} [\Phi(f, g)](u)h_\lambda^*(u) =
  \sum_{u \in \lambda} f(u)h_\lambda(u) + 
  \sum_{u \in \ZZ_+^2} g(u)h_\emptyset^*(u) =
  |f| + |g|
\]
as desired.
\subsection{Binary Sequence Coding for Partitions}
The simplest proof we have seen for this problem involves coding a
partition $\lambda$ as an infinite binary sequence
\[ C_\lambda = \cdots c_{-2}c_{-1}c_0c_1c_2 \cdots \]
This is presented as a solution to exercise 7.107(a) in Stanley, 
The encoding is defined in the following way: Consider the outline of
the Young diagram of $\lambda$, extended to infinity on either side.
Mark any vertical segment with a $0$ and any horizontal segment with a
$1$, and follow the path from the bottom up to the right.  Thus, the
empty partition has the sequence
\[ C_\emptyset = \cdots 00001111 \cdots \]
The partition $\lambda = (1)$ has the sequence
\[ C_{(1)} = \cdots 00010111 \cdots \]
and the partition $\lambda = (3, 1)$ has the sequence
\[ C_{(3, 1)}\cdots 00010110111 \cdots \]
Note that shifting the sequence by any amount defines the same
partition, and beyond that the correspondence is bijective.  Also note
that the sequence always begins with an infinite string of 0's, and
ends with an infinite string of 1's, with only a finite amount of
``interesting'' things happening in the middle.

Now, any hook of lenght $p$ on the \emph{inside} of $\lambda$
corresponds to a strip of length $p$ on the border.  Further, this
corresponds to having $c_i = 1$ and $c_{i+p} = 0$ for some $i$.
Similarly, a hook of lenght $p$ \emph{outside} of $\lambda$
corresponds to having $c_i = 0$ and $c_{i+p} = 1$.  (draw the
picture!)  It should make sense that the former happens only finitely
many times, and the latter happens infinitely many times (although for
a fixed $p$, both still happen only a finite number of times).

Now, we will show that for fixed $p$, there are $p$ more outer hooks
than inner hooks.  Decompose $C_\lambda$ into
\[ C^j_\lambda = \cdots c_{-2p+j}c_{-p+j}c_jc_{p+j}c_{2p+j}\cdots \]
for $0 \leq j < p$.  Then any inner hook is the subsequence $10$
inside some $C^j$, and any outer hook is the subsequence $01$.  Since
each $C^j$ begins with 0's and ends with 1's, there must be exactly
one more $01$ than $10$ in each.  Thus there are $p$ more outer hooks
than inner hooks.

\subsection{Generating Function Proof}
I found this proof while I was trying to formalize by proof by
induction for hook lengths.  It uses Lemma 7.21.1 in Stanley, which
gets at a similar idea of finding hooks by starting with sequences $1,
\cdots n$, and skipping numbers at each corner of the partition.
\begin{lemma}
  Let $\lambda = (\lambda_1, \cdots, \lambda_n)$ be a partition and
  let $\mu_i = \lambda_i + n - i$.  Then
  \[
    \prod_{u \in \lambda} [h(u)] =
    \frac{\prod_{1 \leq i \leq n} [\mu_i]!}{
      \prod_{1\leq i<j\leq n} [\mu_j - \mu_i]}
  \]
  where $[k] = 1-q^k$ and $[k]! = [1][2]\cdots[k]$.
\end{lemma}
Using this notation, what we would like to state is
\[
  \prod_{u\notin\lambda} [h^*(u)] = \prod_{k\geq 1}[k]^k
  \prod_{u\in\lambda} [h(u)]
\]
Define $\nu$ to be the $N\times N$ box with $\lambda$ removed from the
lower-right corner, so $\nu_i = N - \lambda_{N-i+1}$.  (Using the
convention that $\lambda_i = 0$ if $i > n$.)  Define $\rho$ in a way
analagous to $\mu$, so that $\rho_i = N - \nu_i - i = 2N -
\lambda_{N-i+1} - i$.  Thus
\[
  \prod_{u\in \nu} [h(u)] =
  \frac{
    \prod_{1\leq i\leq N} [\rho_i]!
  }{
    \prod_{1\leq i < j \leq N} [\rho_i - \rho_j]
  }
\]
Decompose the product as
\[
  \left(
    \frac{\displaystyle
      \prod_{1\leq i\leq N-n}[\rho_i]!
    }{\displaystyle
      \prod_{1 \leq i < j \leq N-n}[\rho_i-\rho_j]
    }
  \right)\left(
    \frac{\displaystyle
      \prod_{N-n < i \leq N}[\rho_i]!
    }{\displaystyle
      \prod_{1 \leq i \leq N-n < j \leq N}[\rho_i-\rho_j]
    }
  \right)\left(
    \frac{1}{\displaystyle\prod_{N-n<i<j\leq N}[\rho_i - \rho_j]}
  \right)
\]
For $i \leq N - n$, $\rho_i = 2N - i$, so the first term becomes
\[
  \frac{\displaystyle
    \prod_{1\leq i\leq N-n}[\rho_i]!
  }{\displaystyle
    \prod_{1 \leq i < j \leq N-n}[\rho_i-\rho_j]
  } =
  \frac{\displaystyle
    \prod_{1\leq i\leq N-n}[2N - i]!
  }{\displaystyle
    \prod_{1 \leq i < j \leq N-n}[j-i]
  } =
  \frac{\displaystyle
    \prod_{1\leq k < 2N}[k]^{\min(N-n, 2N-k)}
  }{\displaystyle
    \prod_{1 \leq k \leq N-n}[k]^{N-n-k}
  }
  = \prod_{1 \leq k \leq N-n} [k]^k + O(q^{N-n+1})
\]
Where the second equality comes from counting the factors in the
factorial differently.  If we take the limit $N \to \infty$, this term
becomes $\prod_{k \geq 1} [k]^k$.  For the second term, use the fact
that when $i > N - n$, we have $\rho_i = 2N - \lambda_{N-i+1} - i$, or
using the change of variables $k = N - i + 1$, $\rho_{N-k+1} = N -
\lambda_k + k - 1 = N - n - 1 - \mu_k$.  Thus
\begin{align*}
  \frac{\displaystyle
    \prod_{N-n < i \leq N}[\rho_i]!
  }{\displaystyle
    \prod_{1 \leq i \leq N-n < j \leq N}[\rho_i-\rho_j]
  }
  &= \frac{ \displaystyle
    \prod_{1 \leq k \leq n} [N - n - 1 - \mu_k]!
  }{\displaystyle
    \prod_{\substack{1 \leq i \leq N-n \\ 1 \leq k \leq n}} [(2N-i) -
    (N-n-1-\mu_k)]
  } \\
  &= \prod_{1\leq k \leq n} \frac{
    [N-n-1-\mu_k]!
  }{
    [\mu_k+1][\mu_k+1]\cdots [\mu_k + N + n]
    } \\
  &= \prod_{1\leq k \leq n} \frac{[\mu_k]!}{[N-n-\mu_k]\cdots
    [N+n+\mu_k]} \\
  &= \prod_{1\leq k \leq n} [\mu_k]! + O(q^{N-n-\mu_1})
\end{align*}
Note that in the numerator, I use the substitution $k = N - i + 1$,
but in the denominator I use $k = N - j + 1$, before merging the
products.  Also, we must assume $N > n + \mu_1$, which is fine since
$N$ can be arbitrarily large.  As $N \to \infty$, this term becomes
$\prod_{1\leq k\leq n}[\mu_k]!$.

For the last term, use the substitutions $k = N - i + 1$ and $l = N -
j + 1$, so that
\[
  \frac{1}{\displaystyle\prod_{N-n<i<j\leq N}[\rho_i - \rho_j]} =
  \frac{1}{\displaystyle\prod_{1\leq l<k\leq n}
    [\mu_l - \mu_k]}
\]

Thus,
\[
  \prod_{u \notin \lambda} [h^*(u)] =
  \lim_{N\to\infty} \prod_{u \in \nu} [h(u)] =
  \Bigg(\prod_{k\geq 1}[k]^k\Bigg)\Bigg( \prod_{u \in \lambda} [h(u)]\Bigg)
\]

Originally, I was going to formalize my proof by induction by writing
the terms that get added and removed to the generating function when a
new block is added or removed.  This proof is similar in that the
lemma in Stanley is proved using a similar trick.

\subsection{Induction on Cells of $\lambda$}
This proof is by induction on the cells of $\lambda$.  In other words,
the bijection formed is dependant on a SYT of shape $\lambda$, to
decide what order to add cells.  For $\lambda = \emptyset$, the
bijection exists, since $\dt{\emptyset} = \dt{\emptyset}$.

Notice that when one adds a cell to $\lambda$, the only hook lengths
changed are the inner hooks directly above and to the left, and the
outer hooks lengths directly below and to the right.  For the most
part, the hooks are just shifted over by 1, with a finite number of
exceptions.  I claim that these exceptions are exactly equal on the
inside and on the outside.  More specifically, the hooks added and
removed on the bottom equal the ones on the left, and similarly for
the right and top.

For the outer hooks on the bottom, let $a_1$ and $b_1$ be the first
pair of non-consecutive hook lengths, corresponding to the first time
$\lambda_i \neq \lambda_{i+1}$ below the added cell.  Let $c_1$ be
start of the first consecutive sequence on the left, not counting the
one starting from where the cell is added.  Let $d_1$ be end of this
sequence.  Define $a_i, b_i, c_i, d_i$ in terms of the following
sequences.  Define $a_i', b_i', c_i', d_i'$ to be the new hooks after
adding the cell.  Note that $a_i$ and $c_i$ are removed, and $b_i'$
and $d_i'$ are added.  By drawing boxes on the diagram, one can see
that these are always equal.

The case of 1's being added and removed around the added cell is
handled seperately.  There are only a few cases to do by hand, and
they all work out.  This also introduces some arbitrary choices into
how to build the map.
\end{document}